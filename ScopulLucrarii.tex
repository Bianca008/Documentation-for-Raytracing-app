	\^{I}n ultimele decenii, au fost dezvoltate multiple modele ce calculeaz\u{a} modul de propagare al undelor acustice \^{i}n spa\c{t}iul virtual. Din acest motiv am ales s\u{a} dezvolt o aplica\c{t}ie care modeleaz\u{a} propagarea sunetului.
	\bigskip
	
	Exist\u{a} multiple motive pentru dezvoltarea și îmbunătățirea model\u{a}rii acustice \^{i}n \^{i}nc\u{a}peri. Aceast\u{a} lucrare va prezenta ce presupune implementarea unui model acustic, at\^{a}t din punct de vedere fizic, c\^{a}t \c{s}i din punct de vedere geometric \c{s}i vizual. Mai mult de at\^{a}t, va fi prezentat\u{a} \c{s}i implementarea unui model de acest gen, dar \c{s}i rezultatele ob\c{t}inute. Evident, nici un model care simuleaz\u{a} propagarea sunetului \^{i}ntr-un mediu virtual nu va imita 1 la 1 realitatea. Totu\c{s}i, se \^{i}ncearc\u{a} g\u{a}sirea unui algoritm c\^{a}t mai aproape de realitate.
	\bigskip


!aici nu este nici de cum finalizat, ma voi ocupa la final de aceasta bucata!