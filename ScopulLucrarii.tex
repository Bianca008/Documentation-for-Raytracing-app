	\^{I}n ultimele decenii, au fost dezvoltate multiple modele ce calculeaz\u{a} modul de propagare al undelor acustice \^{i}n spa\c{t}iul virtual. Din acest motiv, am ales s\u{a} dezvolt o aplica\c{t}ie software care modeleaz\u{a} propagarea sunetului în încăperi folosind metoda Ray-Tracing. Exist\u{a} multiple motive pentru dezvoltarea și îmbunătățirea model\u{a}rii acustice \^{i}n \^{i}nc\u{a}peri. Aceast\u{a} lucrare va prezenta ce presupune implementarea unui model acustic, at\^{a}t din punct de vedere fizic, c\^{a}t \c{s}i din punct de vedere geometric \c{s}i vizual. Mai mult de at\^{a}t, va fi prezentat\u{a} \c{s}i implementarea unui model de acest tip, dar \c{s}i rezultatele ob\c{t}inute, însoțite de o serie de experimente și o validare a modelului acustic. Evident, nici un model care promovează propagarea sunetului \^{i}ntr-un mediu virtual nu va imita fidel realitatea. Totu\c{s}i, se \^{i}ncearc\u{a} g\u{a}sirea unui algoritm c\^{a}t mai fezabil.

	Atunci când vorbim despre sunetul din cadrul unei încăperi, poate una dintre cele mai importante probleme este să ne dăm seama unde ar trebui să poziționăm sursa audio pentru ca sunetul să fie auzit peste tot și la o calitate cât mai bună. Pentru inginerii acustici propagarea sunetului într-o încăpere este un subiect foarte important, întrucât aceștia au nevoie să știe cum trebuie să realizeze încăperi precum hale, mall-uri, aeroporturi astfel încât acestea să se afle în parametrii acustici pentru a nu deteriora sănătatea urechii omului ce lucrează în medii expuse. De asemenea, atunci când inginerul acustic realizează o biserică sau o catedrală, acesta este nevoit să țină cont de problema propagării sunetul astfel încât acesta să fie auzit peste tot și clar, chiar dacă formele din interiorul acestora variază foarte mult.
	
	În ziua de azi, industria jocurilor a început să se dezvolte foarte mult datorită rolului esențial pe care aceasta îl ocupă în societate. Din acest motiv, au fost realizate o serie de proiecte inovative menite să dezvolte și să promoveze importanța propagării sunetului în jocuri și realitate mixtă, problemă care ar putea fi rezolvată folosind o soluție precum cea promovată de această lucrare.
	
	Pe tema propagării sunetului în încăperi au fost realizate o mulțime de studii pentru a putea găsi un mod cât mai eficient și relevant de a rezolva această problemă. Fiind un subiect de mare interes în timpurile noastre am ales să dezvolt un model acustic menit să simuleze propagarea sunetului în spații interioare folosind metoda Ray-Tracing. Lucrarea urmărește să prezinte modul în care a fost realizat modelul acustic, o serie de rezultate a unor experimente și validarea modelului folosind software-ul Simcenter 3D.
	
	Considerând aceste arii de interes, industria ingineriei acustice și industria jocurilor, am dorit să creez prin această lucrare un model acustic care să simuleze propagarea sunetului în încăperi pentru a veni în ajutorul celor care activează în cele două industrii.