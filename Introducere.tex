Acustica este știința preocupată de producția, controlul, transmisia, recepția și efectele sunetului. Acest termen provine din limba greacă, de la cuvântul \textit{,,akoustos''}. Începând cu originile sale în studiul vibrațiilor mecanice și al radiației acestor vibrații prin unde mecanice, acustica este implicată în aproape toate domeniile vieții. A fost esențială în dezvoltarea culturii popoarele prin crearea instrumentelor muzicale.

Originea științei acusticii este atribuită, în general, filosofului grec Pitagora, ale cărui experimente asupra proprietăților corzilor vibrante care produc intervale muzicale plăcute au avut un merit atât de mare încât au condus la un sistem de acordare care îi poartă numele. Aristotel a sugerat corect că o undă sonoră se propagă în aer prin mișcarea aerului - o ipoteză bazată mai mult pe filosofie decât pe fizica experimentală; totuși, el a sugerat în mod incorect că frecvențele înalte se propagă mai repede decât frecvențele joase - o eroare care a persistat timp de multe secole \cite{istorie}.

Considerând toate acestea, ne putem da seama că omul a acordat permanent o importa\-nță deosebită domeniului acusticii. Întorși în zilele noastre, ne dăm seama că elaborarea unui model acustic este și astăzi un subiect de mare interes.

Urechea umană este sensibilă la vibrațiile aerului cu frecvențe între 20 Hz și 20 kHz și, odată cu vârsta, acest interval se restrânge. Modul în care percepem sunetul diferă de la o încăpere la alta și acest lucru se întâmplă datorită camerelor care au o anumită dimensiune și formă, sunt construite din diverse materiale și conțin suprafețe diferite.

Când discutăm despre o cameră, un aspect foarte important de reținut este unde ar trebui să poziționăm sursele audio și microfoanele pentru a obține cea mai bună calitate a sunetului și pentru a reduce cât mai mult posibil efectul de ecou și de reverberație. În interiorul unei camere, undele sonore lovesc diferite suprafețe care pot absorbi sau reflecta sunetul. Este foarte important pentru inginerii care proiectează modele acustice să aleagă materialele potrivite pentru a obține rezultate optime. De exemplu, fabricile și halele au multe suprafețe metalice care favorizează efectul de ecou. Acest fenomen poate fi verificat de inginerii care proiectează spațiile pentru a se asigura că încăperea respectă toate standardele, iar sănătatea persoanelor care lucrează în acele spații nu este compromisă.

Un model acustic va implica, în cel mai generic mod, simularea căilor pe care sunetul le parcurge de la sursă la destinație. Cel mai adesea, aceste modele propun rezolvarea integralei Helmoltz-Kirchoff \cite{kirchoff} utilizând diverse abordări de calcul, cum ar fi: soluții numerice la ecuațiile de undă, aproximări de frecvență înaltă la ecuația de undă și modele statistice bazate perceptiv. Modelul pe care l-am creat face parte din a doua categorie.

De obicei, termenul de Ray-Tracing este folosit pentru lumină, dar poate fi folosit și pentru sunet, întrucât atunci când avem frecvențe mari, undele sonore au o amplitudine foarte mică și deci le putem aproxima folosind raze. Din acest motiv, am folosit metoda Ray-Tracing pentru a crea un model acustic.

Această lucrare va presupune elaborarea unui model acustic pentru spațiile interioare folosind metoda Ray-Tracing, model care va conține patru etape principale: calcularea geometriei încăperii, realizarea calculelor fizice, post-procesarea datelor obținute și realizarea unei interfețe care să se ocupe de vizualizarea simulării acustice și a rezultatelor obținute de către model.

Pentru a putea evalua corectitudinea modelului au fost realizate două încăperi rectangulare și una sferică folosind platforma Unity. În interiorul acestora au fost plasate o serie de microfoane și o sursă audio. De asemenea, studiul propune un GUI (Graphical User Interface) ușor de folosit pentru a seta configurația dorită pentru fiecare încăpere și pentru a putea vizualiza rezultatele obținute de către modelul acustic. Mai mult de atât, ca să putem valida corectitudinea soluției am folosit software-ul Simcenter3D, o platformă de simulare complet integrată pentru modelarea, simularea și analizarea produselor și sistemelor complexe de inginerie.

Modelul acustic propus de această lucrare va presupune parcurgerea unor etape secven\-țiale, unde output-ul unei etape va reprezenta input-ul următoarei etape. Acesta va conține etapa de calcul geometric ce se va ocupa de simularea razelor și de modul în care acestea vor fi distribuite în încăpere, al doilea pas se va referi la calculele fizice, iar mai apoi va avea loc o etapă de post-procesare a datelor pentru a putea analiza rezultatele fizice obținute pentru fiecare microfon din încăpere.

Ideea acestei lucrări a pornit de la proiectul Triton oferit de Microsoft ce propune rezolvarea problemei propagării sunetului în jocuri și realitate mixtă. Acesta modelează fizic modul în care sunetul se propagă într-o scenă, având în vedere forma și materialele sale. Procedând astfel, modelează automat efecte imersive de propagare a sunetului precum ocluzia și reverberația sunetului. Proiectul Triton este unic în modelarea cu acuratețe a adevăratei fizice a undelor de sunet, inclusiv a difracției. Incubat de peste un deceniu de cercetări concentrate, este o tehnologie testată în luptă, livrată în titluri majore de jocuri precum Gears of War, Sea of Thieves și Borderlands 3. Proiectul Triton modelează fizica reală a undelor de propagare a sunetului prin spații 3D complexe. Sunetele sonore au lungimi de undă de la centimetri la metri, astfel încât efectele de undă trebuie modelate pentru a evita rezultate nenaturale \cite{triton}.

O altă sursă de inspirație a fost lucrarea lui David Oliva Elorza care vorbește despre modelarea acustică în încăperi folosind metoda Ray-Tracing. Autorul prezintă în studiul său noțiunile teoretice de care are nevoie un cititor specializat atunci când dorește să realizeze un model acustic, propune implementarea acestuia și o serie de evaluări ale modelului. Acesta discută o introducere generală a principiilor de propagare a sunetului și o discuție despre stadiul tehnicii în modelarea acustică a încăperilor \cite{elorza}. Față de ce propune Elorza în studiul său, această lucrare propune un mod diferit de distribuire, selecție și reducere al razelor.

În lucrarea \textit{Simulare computerizată a acusticii moscheilor și bisericilor bizantine} a lui Christoffer A. Weitze \cite{chris} este propus un model destinat simulărilor acustice în moscheele și bisericile bizantine, pe când în cadrul acestei lucrări este propus un model acustic mult mai generic.

Astfel, în capitolul ce urmează vor fi prezentate noțiunile teoretice necesare pentru a realiza un model acustic, după care va fi un capitol ce prezintă tehnologiile folosite, urmat de prezentarea modelului acustic implementat și o serie de experimente realizate folosind acest model. Penultimul capitol va conține validarea acestuia folosind software-ul Simcenter 3D, urmat de capitolul final unde se vor prezenta concluziile.
