aici mi-am propus putin sa prezint:

-> ce presupune realizarea unui model acustic, care ar fi etapele importante pe care urmeaza sa le dezvolt(geometric, fizic, post-procesare)

-> de unde a plecat idee, trimitere catre proiect Triton realizat de microsoft, acea lucrare de licenta din italia din care m-am inspirat, articol legat de acustica in moschee si o sa mai caut si altele  

-> sa povestesc un pic si depsre ray-tracing ca este folosit pentru lumina de obicei dar in contextul acestei aplicatii este foarte util pentru ca in acustica, atunci cand avem frecvente mari undele au o amplitudine foarte mica si deci pot fi aproximate folosind raze


