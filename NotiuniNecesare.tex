	\section{Sunetul}
	Sunetul reprezint\u{a} vibra\c{t}ia undelor propagate \^{i}ntr-un mediu, fie el gazos, lichid sau solid \c{s}i prezint\u{a}, \^{i}n multe aspecte, dar nu \^{i}n toate, comportament similar cu alte mi\c{s}c\u{a}ri de und\u{a} pe care le \^{i}nt\^{a}lnim \^{i}n natur\u{a}, adic\u{a} undele de ap\u{a} \c{s}i undele de lumin\u{a}, ale c\u{a}ror fenomene de propagare sunt u\c{s}or de
	observat.
	\bigskip
	
	\^{I}n contextul propag\u{a}rii sunetului, aerul este mediul de interes despre care vom discuta atunci c\^{a}nd vorbim despre acustica \^{i}nc\u{a}perilor, iar perturbarea reprezint\u{a} o alterare a presiunii atmosferice peste \c{s}i sub valoarea sa medie, care produce o mi\c{s}care periodic\u{a} a moleculelor de aer \^{i}napoi \c{s}i \^{i}nainte de-a lungul aceleia\c{s}i direc\c{t}ii \^{i}n care se propag\u{a} unda(unde longitudinale).
	\bigskip
	
	Propagarea sunetului prin aer este ilustrat\u{a} în Figura \ref{Fig1}. \^{I}n partea superioar\u{a}
	este ilustrat\u{a} alterarea presiunii atmosferice, \^{i}n timp ce \^{i}n partea inferioar\u{a} este ilustrat\u{a} mi\c{s}carea moleculelor de aer asociate cu propagarea sunetului. Dac\u{a} intensitatea
	sunetului cre\c{s}te, gradientul presiunii cre\c{s}te \c{s}i, ca urmare, mai multe
	molecule de aer se afl\u{a} \^{i}n mi\c{s}care.
	
	\begin{figure}[!htb]
		\centering
		\includegraphics[width=12cm]{imagini/propagareaSunetuluiInAer.png}
		\caption{Propagarea sunetului prin aer\cite{elorza}}
		\label{Fig1}
	\end{figure}

	Un principiu general, stabilit pentru prima dat\u{a} de Fermat, afirm\u{a} c\u{a} fiecare und\u{a} se propag\u{a} de la surs\u{a} c\u{a}tre receptor prin calea cea mai rapid\u{a}. Dac\u{a} mediul este omogen, precum consider\u{a}m c\u{a} este aerul, atunci viteza sunetului este uniform\u{a} prin acest mediu \c{s}i astfel calea cea mai rapid\u{a} devine totodat\u{a} \c{s}i cea mai scurt\u{a}.
	Viteza sunetului depinde doar de temperatur\u{a}, nu \c{s}i de alte propriet\u{a}\c{t}i precum presiunea \c{s}i densitatea. Astfel, putem observa \^{i}n Tabelul \ref{Tabel1} cum variaz\u{a} viteza \c{s}i densitatea sunetului \^{i}n func\c{t}ie de temperatur\u{a}.
	
	\bigskip
	\begin{table}[!htb]
		\centering		
	\begin{tabular}{|c|c|c|}
		\hline
		\bf{Temperatur\u{a}($^{\circ}$C)} & \bf{Viteza sunetului$\left(\dfrac{m}{s} \right)$} & \bf{Densitatea aerului$\left(\dfrac{kg}{m^3}\right)$}\\
		\hline \hline
		
		30 & 349.02 & 1.1644\\
		\hline
		25 & 346.13 & 1.1839\\
		\hline
		20 & 343.21 & 1.2041\\
		\hline
		15 & 340.27 & 1.2250\\
		\hline
		10 & 337.31 & 1.2466\\
		\hline
		5 & 334.32 & 1.2690\\
		\hline
		0 & 331.30 & 1.2922\\
		\hline			
	\end{tabular}
	\caption{Efectul temperaturii asupra propriet\u{a}\c{t}ilor aerului\cite{temperaturaTabel}}
	\label{Tabel1}
	\end{table}

	{\it{Amplitudinea}} este valoarea absolut\u{a}, maxim\u{a}, a unei cantit\u{a}\c{t}i care variaz\u{a} periodic. Amplitudinile sunt exprimate fie ca valori instantanee, fie mai ales ca valori de v\^{a}rf. Aceasta reprezint\u{a} fluctua\c{t}ia sau deplasarea unei unde de la valoarea sa medie. \^{I}n cazul undelor sonore, particulele de aer sunt deplasate, iar aceast\u{a} amplitudine a sunetului este exprimat\u{a} ca intensitatea sunetului. Amplitudinea nu este influen\c{t}at\u{a} de frecven\c{t}\u{a}, de lungimea de und\u{a}, de perioada de timp sau de viteza sunetului \c{s}i nici invers.
	\bigskip

	{\it{Lungimea de und\u{a}}} este reprezentat\u{a} de distan\c{t}a dintre punctele consecutive corespunz\u{a}toare ale aceleia\c{s}i faze de und\u{a}, precum dou\u{a} creste adiacente. {\it{Viteza de propagare($c$)}} a undei este dat\u{a} de lungimea de und\u{a}, pe care o vom nota cu $\lambda$, fiind distan\c{t}a parcurs\u{a} de val \^{i}ntre dou\u{a} instan\c{t}e de faz\u{a} egal\u{a} \c{s}i de timp, $T$, timpul necesar acestei distan\c{t}e.
	
	\begin{equation}
	c=\lambda/T
	\end{equation}
	\bigskip
	
	Inversa perioadei este numit\u{a} {\it{frecven\c{t}\u{a}($f$)}} \c{s}i indic\u{a} de c\^{a}te ori particulele de aer s-au mi\c{s}cat \^{i}napoi \c{s}i \^{i}nainte \^{i}ntr-o secund\u{a}. Frecven\c{t}a este m\u{a}surat\u{a} \^{i}n Hertz[Hz].
	
	\begin{equation}
	c=\lambda f
	\end{equation}
	\bigskip
	
	O caracteristic\u{a} important\u{a} a unei unde sonore este {\it{faza}}, care specific\u{a} loca\c{t}ia unui punct \^{i}n cadrul unui ciclu de und\u{a} al unei forme de und\u{a} repetitive. \^{I}n majoritatea cazurilor, diferen\c{t}ele de faz\u{a} dintre undele sonore sunt mai importante dec\^{a} fazele \^{i}n sine. Diferen\c{t}a de faz\u{a} dintre dou\u{a} unde sonore cu aceea\c{s}i frecven\c{t}\u{a} care se deplaseaz\u{a} dincolo de o loca\c{t}ie fix\u{a} este dat\u{a} de diferen\c{t}a de timp dintre acelea\c{s}i pozi\c{t}ii \^{i}n cadrul ciclurilor de und\u{a} ale celor dou\u{a} sunete, exprimate ca o frac\c{t}iune dintr-un ciclu de und\u{a}. 
	\bigskip 
	
	Dou\u{a} unde sonore de aceea\c{s}i frecven\c{t}\u{a} care sunt perfect aliniate au o diferen\c{t}\u{a} de faz\u{a} de 0 \c{s}i se spune c\u{a} sunt ,,\^{i}n faz\u{a}". Dou\u{a} unde care sunt ,,\^{i}n faz\u{a}" se adaug\u{a} pentru a produce o und\u{a} sonor\u{a} cu o amplitudine egal\u{a} cu suma amplitudinilor celor dou\u{a} unde.

	
	\begin{figure}[!htb]
		\centering
		\includegraphics[width=10cm]{imagini/soundWaveGreaterAmplitude.png}
		\caption{Dou\u{a} unde sonore ,,\^{i}n faz\u{a}"}
		\label{Fig7}
	\end{figure}

	Dac\u{a} una dintre cele dou\u{a} unde sonore cu aceea\c{s}i frecven\c{t}\u{a} este deplasat\u{a} cu o jum\u{a}tate de ciclu fa\c{t}\u{a} de cealalt\u{a}, se spune c\u{a} undele sonore sunt ,,defazate". Dou\u{a} unde sunt ,,defazate" dac\u{a} se anuleaz\u{a} reciproc exact c\^{a}nd sunt adunate \^{i}mpreun\u{a}.

	\begin{figure}[!htb]
		\centering
		\includegraphics[width=10cm]{imagini/soundWaveCanceledAmplitude.png}
		\caption{Dou\u{a} unde sonore ,,defazate"}
		\label{Fig8}
	\end{figure}

	Diferen\c{t}a de faz\u{a} este exprimat\u{a} ca un unghi, deoarece forma de und\u{a} a unui ton pur alc\u{a}tuit\u{a} dintr-o singur\u{a} frecven\c{t}\u{a} poate fi descris\u{a} utiliz\^{a}nd func\c{t}ia sinus trigonometric\u{a}.
	\bigskip
	
	Cele mai multe sunete sunt mult mai complexe dec\^{a}t o singur\u{a} frecven\c{t}\u{a}, dar constau \^{i}n schimb din multe unde sinusoidale diferite la frecven\c{t}e diferite. C\^{a}nd mai multe unde sinusoidale se combin\u{a} pentru a crea un sunet, formele de und\u{a} ale tuturor undelor sinusoidale sunt ad\u{a}ugate la fiecare loca\c{t}ie de-a lungul formei de und\u{a}.
	\bigskip
	
	{\it{Energia(W)}} poate fi auzit\u{a} de fiin\c{t}ele vii. Sunetul este o und\u{a} mecanic\u{a} \c{s}i ca atare const\u{a} fizic \^{i}n compresie elastic\u{a} oscilatorie \c{s}i \^{i}n deplasarea oscilatorie a unui fluid. Prin urmare, mediul ac\c{t}ioneaz\u{a} ca stocare at\^{a}t pentru energia poten\c{t}ial\u{a}, c\^{a}t \c{s}i pentru energia cinetic\u{a}. \^{I}n consecin\c{t}\u{a}, energia sonor\u{a} dintr-un volum de interes este definit\u{a} ca suma densit\u{a}\c{t}ilor de energie poten\c{t}ial\u{a} \c{s}i cinetic\u{a}:
	
	\begin{equation}
	W = W_{\text{poten\c{t}ial\u{a}}} + W_{\text{cinetic\u{a}}}
	\end{equation} 
	\bigskip
	
	{\it{Presiunea acustic\u{a}}} este abaterea presiunii locale fa\c{t}\u{a} de presiunea atmosferic\u{a}. \^{I}n aer, presiunea poate fi m\u{a}surat\u{a} cu ajutorul unui microfon, iar \^{i}n ap\u{a} cu ajutorul unui hidrofon. Unitatea de m\u{a}sur\u{a} dat\u{a} de c\u{a}tre Sistemul interna\c{t}ional de unit\u{a}\c{t}i de m\u{a}sur\u{a}, SI, pentru presiunea acustic\u{a} este Pascal(Pa). Formula folosit\u{a} pentru transformarea intensit\u{a}\c{t}ii \^{i}n presiune este:
	
	\begin{equation}
	p = \sqrt{2 I \rho c}, 
	\end{equation}
	
	\noindent unde $\rho$ este densitatea aerului, iar $c$ este viteza sunetului.
	\bigskip
	
	{\it{Puterea sunetului}} este rata la care energia sonor\u{a} este emis\u{a}, reflectat\u{a}, transmis\u{a} sau recep\c{t}ionat\u{a}, pe unitate de timp. Unitatea SI pentru puterea sunetului este Watt-ul(W). Pentru o surs\u{a} de sunet, spre deosebire de presiunea sonor\u{a}, puterea nu este dependent\u{a} nici de \^{i}nc\u{a}pere, nici de distan\c{t}\u{a}. Presiunea sonor\u{a} este o proprietate a c\^{a}mpului \^{i}ntr-un punct din spa\c{t}iu, \^{i}n timp ce puterea sonor\u{a} este proprietatea unei surse sonore, egal\u{a} cu puterea total\u{a} emis\u{a} de acea surs\u{a} \^{i}n toate direc\c{t}iile.
	\bigskip
	
	{\it{Intensitatea(I)}} este definit\u{a} ca putere pe unitate de suprafa\c{t}\u{a} purtat\u{a} de o und\u{a}. {\it{Puterea(P)}} este rata la care energia este transferat\u{a} de und\u{a}. Unitatea de m\u{a}sur\u{a} folosi\u{a} pentru intensitate este $\dfrac{W}{m^2}$, iar formula acesteia este:
	
	\begin{equation}
	I=\frac{P}{A}
	\end{equation}
	\bigskip
	
	{\it{Nivelurile de intensitate}} ale sunetului sunt citate în decibeli (dB). Modul în care urechile noastre percep sunetul poate fi descris mai exact prin logaritmul intensit\u{a}\c{t}ii. Nivelul de intensitate($\beta$) este definit astfel:
	
	\begin{equation}
	\beta(dB) = 10 \lg\left(\dfrac{I}{I_0}\right) 
	\end{equation}
	
	\section{Reflexia sunetului}
	
	Principiile reflexiei pot fi aplicate undelor sonore, care constau din compresii \c{s}i refrac\c{t}ii. Dac\u{a} o und\u{a} sonor\u{a} se deplaseaz\u{a} printr-un tub cilindric, \^{i}n cele din urm\u{a} sunetul va ajunge la cap\u{a}tul tubului, acesta reprezent\^{a}nd grani\c{t}a dintre aerul din tub \c{s}i aerul din afara tubului. La atingerea cap\u{a}tului tubului, unda sonor\u{a} va suferi o reflec\c{t}ie par\c{t}ial\u{a}(o parte din energia transportat\u{a} va r\u{a}m\^{a}ne \^{i}n tub \c{s}i se va deplasa \^{i}n direc\c{t}ia opus\u{a}) \c{s}i o transmisie par\c{t}ial\u{a}(o parte din energia transportat\u{a} va trece peste grani\c{t}\u{a}, \^{i}n afara tubului).
	\bigskip
	
	Reflexia pe suprafe\c{t}e poate duce la unul dintre urm\u{a}toarele fenomene: ecou sau reverbera\c{t}ie. {\it{Ecoul}} este o reflexie a sunetului care ajunge la ascult\u{a}tor cu o \^{i}nt\^{a}rziere fa\c{t}\u{a} de sunetul direct. Aceast\u{a} \^{i}nt\^{a}rziere este direct propor\c{t}ional\u{a} cu distan\c{t}a suprafe\c{t}ei reflectate de la surs\u{a} la ascult\u{a}tor. Undele acustice sunt reflectate de pere\c{t}i sau de alte suprafe\c{t}e dure, cum ar fi mun\c{t}ii. Ecoul poate fi auzit atunci c\^{a}nd reflexia revine cu o amplitudine \c{s}i o \^{i}nt\^{a}rziere suficient\u{a} pentru a fi perceput\u{a} distinct. 
	\bigskip
	
	{\it{Reverberarea}} reprezint\u{a} o persisten\c{t}\u{a} a sunetului dup\u{a} ce acesta a fost produs. O reverbera\c{t}ie este creat\u{a} atunci c\^{a}nd un sunet sau semnal provoac\u{a} multiple reflexii care se acumuleaz\u{a} \c{s}i apoi se descompun pe m\u{a}sur\u{a} ce sunetul este absorbit de suprafe\c{t}e. \^{I}n Figura \ref{Fig2}, observ\u{a}m dou\u{a} diagrame cu axa x reprezent\^{a}nd timpul \c{s}i cu axa y reprezent\^{a}nd SPL(Sound Pressure Level) ce ilustreaz\u{a} diferen\c{t}a dintre ecou \c{s}i reverbera\c{t}ie 
	
	\begin{figure}[!htb]
		\centering
		\includegraphics[width=12cm]{imagini/EchoReverberation.jpg}
		\caption{Difere\c{t}a dintre ecou \c{s}i reverbera\c{t}ie}
		\label{Fig2}
	\end{figure}

	Trasarea razelor pentru calcularea unghiurilor de inciden\c{t}\u{a} sau refrac\c{t}ie este dat\u{a} de legea lui Snell prin Ecua\c{t}ia \eqref{SnellEq}(vezi \cite{snell}):
	
	\begin{equation}
		\label{SnellEq}
		\frac{\sin \theta_2}{\sin \theta_1} = \frac{v_2}{v_1} = \frac{n_1}{n_2}
	\end{equation}

	\noindent unde $\theta_1, \theta_2$ sunt unghiurile de refrac\c{t}ie, $v_1, v_2$ reprezint\u{a} viteza de propagare prin mediu, iar $n_1, n_2$ sunt indicii de refrac\c{t}ie din mediu.
	\bigskip

	Reflexiile sunetului pe suprafe\c{t}e nu urmeaz\u{a} \^{i}ntotdeauna legile lui Snell. Astfel, nici o reflexie nu este perfect specular\u{a}, deci este par\c{t}ial difuz\u{a}, iar acest lucru se \^{i}nt\^{a}mpl\u{a} ca o consecin\c{t}\u{a} pentru duritatea \c{s}i dimensiunea suprafe\c{t}ei de coliziune. Când o rază întâlnește o
	suprafață difuză, se generează un număr aleatoriu în intervalul [0, 1]. Dac\u{a} num\u{a}rul este mai mic dec\^{a}t un prag ales direc\c{t}ia razei este randomizat\u{a} pentru a simula difuzia, altfel	reflexia este specular\u{a}. Acest fenomen poate fi eviden\c{t}iat prin Figura \ref{Fig10}. \^{I}n aceast\u{a} lucrare se va folosi reflexia difuz\u{a} prin randomizare normal\u{a}.
	
	\begin{figure}[!htb]
		\centering
		\includegraphics[width=12cm]{imagini/reflections.png}
		\caption{Reflexie difuz\u{a} prin randomizare normal\u{a} \c{s}i reflexie difuz\u{a} prin randomizare ponderat\u{a}}
		\label{Fig10}
	\end{figure}

	\section{Difrac\c{t}ia \c{s}i interferen\c{t}a sunetului}
	
	Difrac\c{t}ia reprezint\u{a} schimbarea local\u{a} \^{i}n direc\c{t}ia propag\u{a}rii undelor sonore trec\^{a}nd de marginea unui obstacol. Acesta este unul dintre cele mai importante fenomene acustice cauzat de natura sunetului, fiind una dintre cele mai complexe probleme de rezolvat. Efectele difrac\c{t}iei pot fi \^{i}mp\u{a}r\c{t}ite \^{i}n trei grupe: barier\u{a}, margine \c{s}i deschidere.
	\bigskip
	
	Fenomenul de difrac\c{t}ie depinde semnificativ de raportul dintre lungimea de und\u{a} a sunetului si m\u{a}rimea obstacolului. Cu c\^{a}t lungimea de und\u{a} este mai mare, cu at\^{a}t sunetul se intensific\u{a}. C\^{a}nd o und\u{a} sonor\u{a} \^{i}nt\^{a}lne\c{s}te un obstacol, care este mic \^{i}n raport cu lungimea de und\u{a}, valul trece \^{i}n jurul ei ca \c{s}i c\^{a}nd nu ar exista, form\^{a}nd foarte pu\c{t}in\u{a} umbr\u{a}. Dac\u{a} frecven\c{t}a sunetului este suficient de mare, lungimea de und\u{a} este suficient de scurt\u{a} \c{s}i ca urmare se formeaza\u{a} o umbr\u{a} vizibil\u{a}.
	\bigskip
	
	\^{I}n Figura \ref{Fig3} se pot observa cele trei tipuri de difrac\c{t}ii, unde situa\c{t}iile din partea de sus sunt ideale pentru frecven\c{t}ele joase, iar cazurile din partea de jos sunt de dorit pentru frecven\c{t}ele \^{i}nalte.
	\bigskip
	
	\begin{figure}[!htb]
		\centering
		\includegraphics[width=12cm]{imagini/difractie.png}
		\caption{Difrac\c{t}ie barier\u{a}, deschidere, margine\cite{elorza}}
		\label{Fig3}
	\end{figure}
	
	Dou\u{a} unde care c\u{a}l\u{a}toresc \^{i}n acela\c{s}i mediu vor interfera una cu cealalt\u{a}. Dac\u{a} amplitudinile lor se adun\u{a}, se spune c\u{a} aceasta este o interferen\c{t}\u{a} constructiv\u{a}. O interferen\c{t}\u{a} distructiv\u{a} se realizeaz\u{a} atunci c\^{a}nd cele dou\u{a} unde sonore sunt defazate \c{s}i scad.
	\bigskip
	
	\^{I}n acest studiu, difrac\c{t}ia \c{s}i interferen\c{t}a nu vor fi abordate.
	
	\section{Func\c{t}ia de r\u{a}spuns la impuls}
	
	Propagarea sunetului de la o surs\u{a} audio la un receptor se caracterizeaz\u{a} prin func\c{t}ia de r\u{a}spuns impuls \c{s}i informa\c{t}iile spa\c{t}iale pe toate c\u{a}ile de propagare posibile. Prima detectare corespunde \^{i}ntotdeauna cu sunetul direct \c{s}i, dup\u{a} acesta, primesc reflexii multiple. \^{I}nt\^{a}rzierile lor de timp \^{i}n ceea ce prive\c{s}te sunetul direct sunt \^{i}n func\c{t}ie de lungimile c\u{a}ilor parcurse \c{s}i intensit\u{a}\c{t}ile lor de presiune depind de absorb\c{t}ia sunetului prin aer, precum \c{s}i de caracteristicile de absorb\c{t}ie a suprafe\c{t}elor implicate \^{i}n fiecare cale.
	\bigskip
	
	R\u{a}spunsul la impuls este func\c{t}ia de ie\c{s}ire a unui sistem dinamic atunci c\^{a}nd la intrare se aplic\u{a} o func\c{t}ie unitar\u{a}(func\c{t}ia Dirac). Un impuls este un eveniment sonor foarte puternic \c{s}i scurt, care este utilizat pentru testarea r\u{a}spunsului la sunet \^{i}ntr-o camer\u{a} sau pentru a testa eficien\c{t}a unui sistem acustic. Un impuls con\c{t}ine toate frecven\c{t}ele.
	\bigskip 
	
	O func\c{t}ie de r\u{a}spuns la impuls se compune din: sunet direct, prima \^{i}nt\^{a}rziere, reflec\c{t}ii timpurii \c{s}i coada reverberant\u{a}. Figura \ref{Fig4} ilustreaz\u{a} componentele unei func\c{t}ii de r\u{a}spuns la impuls.
	\bigskip
	
	\begin{figure}[!htb]
		\centering
		\includegraphics[width=8cm]{imagini/impulseResponse.png}
		\caption{Func\c{t}ia de r\u{a}spuns la impuls}
		\label{Fig4}
	\end{figure}

	{\it{Sunetul direct(Direct Sound)}} are presiunea acustic\u{a} ridicat\u{a}, dar durat\u{a} scurt\u{a}, reprezent\^{a}nd timpul necesar pentru ca sunetul s\u{a} ajung\u{a} la cel primit(ex: ascult\u{a}tor sau microfon).
	\bigskip
	
	{\it{Decalaj ini\c{t}ial de timp(Initial Time Delay Gap)}}	reprezint\u{a} timpul dintre sunetul direct \c{s}i primele reflexii \c{s}i ne spune c\^{a}t de departe este sursa de sunet. Cu c\^{a}t decalajul ini\c{t}ial de timp este mai lung, cu at\^{a}t este mai apropiat\u{a} sursa de sunet. Cu alte cuvinte, dac\u{a} sursa este departe, sunetul direct \c{s}i primele reflexii vor fi auzite mai apropiate.
	\bigskip
	
	{\it{Reflec\c{t}ii timpurii(First Order Reflections sau Early Reflections)}} sunt primele pe care le auzim \c{s}i se disting. Este posibil s\u{a} fie doar c\^{a}teva \^{i}ntr-o camer\u{a} simpl\u{a} dreptunghiular\u{a}, dar pot fi mai multe dac\u{a} camera este mai complex\u{a}. Reflec\c{t}iile timpurii indic\u{a} c\^{a}t de mare este o camer\u{a}.
	\bigskip
	
	{\it{Coada reverberant\u{a}}} const\u{a} \^{i}n reflexii de ordin superior \c{s}i nu se pot distinge \^{i}ntre ele. Pe m\u{a}sur\u{a} ce num\u{a}rul de reflexii cre\c{s}te, undele sonore pierd energie \c{s}i, \^{i}n cele din urm\u{a}, se descompun. Dezintegrarea reverberant\u{a} este adesea liniar\u{a} atunci c\^{a}nd este reprezentat\u{a} grafic ca mai sus. 
	
	\section{Func\c{t}ia de r\u{a}spuns la frecven\c{t}\u{a}}
	
	Răspunsul în frecvență este măsura cantitativă a spectrului de ieșire al unui sistem sau dispozitiv ca răspuns la un stimul și este utilizat pentru a caracteriza dinamica sistemului. Este o măsură a magnitudinii și a fazei în funcție de frecvență, în comparație cu intrarea. În termeni simpli, dacă o undă sinusoidală este injectată într-un sistem la o frecvență dată, un sistem liniar va răspunde la aceeași frecvență cu o anumită magnitudine și un anumit unghi de fază relativ la intrare.
	\bigskip
	
	\^{I}n contextul unui sistem audio, obiectivul poate fi reproducerea semnalului de intrare f\u{a}r\u{a} distorsiuni. Acest lucru necesit\u{a} o amplitudine de r\u{a}spuns uniform p\^{a}n\u{a} la limitarea l\u{a}\c{t}imii de band\u{a} a sistemului, cu semnalul \^{i}nt\^{a}rziat cu exact aceea\c{s}i cantitate de timp la toate frecven\c{t}ele.
	\bigskip
	
	Estimarea r\u{a}spunsului \^{i}n frecven\c{t}\u{a} pentru un sistem fizic implic\u{a}, \^{i}n general, excitarea semnalului cu un semnal de intrare, m\u{a}surarea istoricelor de timp de intrare \c{s}i de ie\c{s}ire \c{s}i compararea celor dou\u{a} printr-un proces precum Transformarea Fourier Rapid\u{a}(FFT).
	\bigskip
	
	R\u{a}spunsul \^{i}n frecven\c{t}\u{a} se caracterizeaz\u{a} prin amploarea r\u{a}spunsului sistemului, m\u{a}surat\u{a}, de obicei, \^{i}n decibeli(dB), \c{s}i faza, m\u{a}surat\u{a} \^{i}n radiani sau grade.
	
	\section{Absorb\c{t}ia sunetului}
	
	\^{I}n contextul propag\u{a}rii sunetului \^{i}n spa\c{t}ii \^{i}nchise vom considera dou\u{a} tipuri de absorb\c{t}ii: \textit{absorb\c{t}ia aerului} \c{s}i \textit{absorb\c{t}ia suprafe\c{t}elor}.
	\bigskip
	
	\textit{Coeficientul de absorb\c{t}ie \^{i}n aer}, depinde de temperatura atmosferic\u{a}, de presiunea atmosferic\u{a} \c{s}i de frecven\c{t}\u{a}. Absorb\c{t}ia sunetului \^{i}n acustic\u{a} este cauzat\u{a} de caracteristicile de absorb\c{t}ie ale suprafe\c{t}elor.
	\bigskip
	
	Astfel, absorb\c{t}ia este definit\u{a} ca o disipare a energiei sonore la lovirea unei suprafe\c{t}e fizice. La fiecare reflexie, o parte $\alpha$ din energia sau puterea sa este absorbit\u{a}. Acest factor $\alpha$ se nume\c{s}te coeficient de absorb\c{t}ie. Absorb\c{t}ia sunetului depinde de unghiul de inciden\c{t}\u{a}.
	\bigskip
	
	Un alt mecanism de absorb\c{t}ie a sunetului este \textit{absorb\c{t}ia suprafe\c{t}elor}, fiind definit ca disiparea energiei sonore la lovirea unei suprafe\c{t}e fizice. La fiecare reflexie, o parte din energie sau putere este absorbit\u{a}. Restul de energie care nu este absorbit\u{a} este fie absorbit\u{a} de material, fie transmis\u{a} mai departe ca sunet reflectat. Formula utilizat\u{a} \^{i}n aceast\u{a} lucrare este eviden\c{t}iat\u{a} prin Ecuatia \eqref{ecuatia3}.
	
	\begin{equation}
		W = W(1-\alpha)
		\label{ecuatia3}
	\end{equation}
	
	\noindent unde $W$ este energia.
	
	\section{Simularea binaural\u{a}}
	
	\textit{Simularea binaural\u{a}} reprezint\u{a} o metod\u{a} de a realiza semnale binaurale la ambele urechi ale receptorului \^{i}n spa\c{t}ii inexistente prin intermediul unui model. \^{I}n ultima perioad\u{a}, simularea acustic\u{a} a devenit din ce \^{i}n ce mai important\u{a} pentru proiectarea spa\c{t}iilor \^{i}nchise, precum s\u{a}lile de concerte \c{s}i de teatru.
	\bigskip
	
	Exist\u{a} o mul\c{t}ime de tehnici de simulare binaural\u{a} \^{i}n combina\c{t}ie cu metode de calcul, de cele mai multe ori tehnici geometrice pentru calcul acustic, precum: metoda sursei de imagine \^{i}n oglind\u{a}, ray-tracing sau metode hibride. Cu toate acestea, genul acesta de tehnici vin la pachet cu cre\c{s}terea timpului de calcul. Unul dintre cele mai dificile lucruri este directivitatea sursei.
	
	\bigskip
	De\c{s}i este imposibil\u{a} simularea cu exactitate a unui c\^{a}mp sonor real folosind metode geometrice, lucr\u{a}rile realizate p\^{a}n\u{a} \^{i}n prezent pe aceste teme s-au bucurat de un mare succes, iar unele au fost chiar comercializate. O alt\u{a} provocare \^{i}n crearea unui astfel de algoritm este dat\u{a} de coada reverberant\u{a} datorit\u{a} efortului considerabil de timp. Simularea binaural\u{a} nu va fi tratat\u{a} \^{i}n acest studiu.
	
	\begin{figure}[!htb]
		\centering
		\includegraphics[width=12cm]{imagini/binauralExample.png}
		\caption{Exemplu de simulare binaural\u{a}}
		\label{Fig12}
	\end{figure}

	\^{I}n Figura \ref{Fig12}, se poate observa beneficiul unei simul\u{a}ri binaurale. \^{I}n prima parte a figurii, sunetul ajunge de la surs\u{a}(S) la urechile omului \^{i}n acela\c{s}i moment \c{s}i cu aceea\c{s}i intensitate, pe c\^{a}nd \^{i}n cea de-a doua parte a imaginii sunetul ajunge mai devreme la urechea st\^{a}ng\u{a} dec\^{a}t la urechea dreapt\u{a} \c{s}i la intensit\u{a}\c{t}i diferite. Urechea st\^{a}ng\u{a} va auzi mai puternic sunetul dec\^{a}t urechea dreapt\u{a}. 

	\section{Tehnica de trasare a razelor(Ray-Tracing)}
	
	Intensitatea emis\u{a} de o surs\u{a} este descris\u{a} de un num\u{a}r finit de raze, care vor fi considerate purt\u{a}tori de intensitate(sau de energie sau de putere). Aceste raze c\u{a}l\u{a}toresc prin spa\c{t}iu la viteza sunetului \c{s}i sunt reflectate dup\u{a} fiecare coliziune cu limitele camerei. \^{I}n acest timp, intensitatea lor scade ca o consecin\c{t}\u{a} a absorb\c{t}iei aerului \c{s}i a pere\c{t}ilor pe care raza \^{i}i intersecteaz\u{a}.
	\bigskip
	
	Dac\u{a} sursa sonor\u{a} are caracteristici omnidirec\c{t}ionale, direc\c{t}iile razelor sunt create prin distribu\c{t}ii aleatoare omogene. Calea posibil\u{a} pentru fiecare raz\u{a} emis\u{a} de la surs\u{a} trebuie gasit\u{a}. Raza este considerat\u{a} un vector care \^{i}\c{s}i schimb\u{a} direc\c{t}ia la fiecare reflexie, fiind astfel posibil\u{a} determinarea urm\u{a}toarei suprafe\c{t}e de ciocnire. 
	\bigskip
	
	De obicei, sursa este \^{i}mp\u{a}r\c{t}it\u{a} \^{i}ntr-un num\u{a}r mare de piese mici. Metoda determinist\u{a} este \^{i}mp\u{a}r\c{t}irea sursei \^{i}ntr-o manier\u{a} matematic\u{a} sau geometric\u{a}. Piesele din surs\u{a} ar trebui s\u{a} fie identice sau c\^{a}t de mult posibil identice. Punctele selectate aleatoriu pe suprafa\c{t}a unei sfere surs\u{a} pot fi reprezentate de trei parametrii: raza, azimutul \c{s}i eleva\c{t}ia\cite{jeong}. Mai mult dec\^{a}t a propus Jeong Cheol-Ho \^{i}n studiul s\u{a}u, \^{i}n aceast\u{a} lucrare \^{i}mp\u{a}r\c{t}irea sursei \^{i}n piese mai mici s-a realizat cu ajutorul sferei lui Fibonacci, algoritm ce urmeaz\u{a} s\u{a} fie descris \^{i}n capitolele urm\u{a}toare.
	\bigskip
	
	Mai departe, aceste puncte reprezint\u{a} pozi\c{t}ia de start pentru o raz\u{a}. Fiecare raz\u{a} poate avea nici una sau mai multe reflexii \c{s}i, astfel, s\u{a} fie considerat\u{a} o raz\u{a} direct\u{a} sau indirect\u{a}. Fiecare coliziune este re\c{t}inut\u{a} \c{s}i putem observa care a fost traseul pe care fiecare raz\u{a} l-a parcurs \^{i}n \^{i}nc\u{a}pere de la surs\u{a} p\^{a}n\u{a} la ascult\u{a}tor.
	\bigskip
	
	Intensitatea pe care o raz\u{a} o transmite receptorului este direct dependent\u{a} de distan\c{t}a pe care raza a parcurs-o \^{i}n \^{i}nc\u{a}pere. Intensitatea la un anumit moment se calculeaz\u{a} astfel dependent de absorb\c{t}ia aerului, de lungimea traseului parcurs \c{s}i de propriet\u{a}\c{t}ile de absorb\c{t}ie ale peretelui. Coliziunea cu suprafa\c{t}a poate fi considerat\u{a} specular\u{a}, conform legii lui Snell.	Tot acest proces trebuie repetat pentru fiecare frecven\c{t}\u{a}.
	\bigskip
	
	\^{I}n Figura \ref{Fig9} este ilustrat un mod de reprezentare pentru tehnica Ray-Tracing, unde se poate observa traiectul razelor. Linia continu\u{a} reprezint\u{a} raza direct\u{a}, iar liniile punctate semnific\u{a} razele indirecte.
	
	\begin{figure}[!htb]
		\centering
		\includegraphics[width=12cm]{imagini/roomExample.png}
		\caption{Exemplu de ray-tracing}
		\label{Fig9}
	\end{figure}
	
	S-a demonstrat c\u{a} algoritmul de Ray-Tracing prezice nivelurile de zgomot cu o precizie foarte bun\u{a} \c{s}i este considerat drept unul dintre cele mai elegante metode de reprezentare a reflexiilor. Cu toate acestea, exist\u{a} limit\u{a}ri, excep\c{t}ii \c{s}i probleme. Exist\u{a} c\^{a}teva elemente care ar trebui luate \^{i}n considerare atunci c\^{a}nd implement\u{a}m sau evalu\u{a}m un astfel de algoritm, precum: \textit{intervalul de frecven\c{t}e}, \textit{factori dependen\c{t}i de frecven\c{t}\u{a}}, \textit{geometria}, \textit{num\u{a}rul de raze}.
	\bigskip
	
	\^{I}n mod normal, singurii factori dependen\c{t}i de frecven\c{t}\u{a} care pot fi inclu\c{s}i sunt coeficien\c{t}ii de absorb\c{t}ie \c{s}i estimarea statistic\u{a} a propriet\u{a}\c{t}ilor difuze. O ipotez\u{a} de baz\u{a} \^{i}n metodele care utilizeaz\u{a} raze este c\u{a} lungimea de und\u{a} corespunz\u{a}toare celei mai mici frecven\c{t}e este mai mic\u{a} \^{i}n compara\c{t}ie cu dimensiunile camerei \c{s}i suprafe\c{t}ele acesteia.
	\bigskip
	
	Precum am men\c{t}ionat mai sus, coeficientul de absorb\c{t}ie $\alpha$ este dependent de unghi. Principalul motiv pentru evitarea m\u{a}sur\u{a}torilor dependente de unghi este c\u{a} acestea ar trebui s\u{a} fie foarte precise. Acest lucru fiind destul de dificil, \^{i}ntruc\^{a}t presupune stocarea tuturor materialelor de construc\c{t}ie. Prin urmare, a fost acceptat\u{a} estimarea caracteristicilor de absorb\c{t}ie a unei suprafe\c{t}e printr-o form\u{a} care este \^{i}n medie peste toate unghiurile de inciden\c{t}\u{a}.
	\bigskip
	
	Erori numerice sunt introduse \^{i}n rezultate prin utilizarea unui num\u{a}r limitat de raze, deoarece unghiul dintre raze adiacente r\u{a}m\^{a}ne constant, iar aceast\u{a} reprezentare devine treptat mai pu\c{t}in exact\u{a} odat\u{a} cu sc\u{a}derea num\u{a}rului de raze. Pentru a atinge criteriile de convergen\c{t}\u{a}, num\u{a}rul de raze trebuie s\u{a} fie c\^{a}t mai mare, cu c\^{a}t num\u{a}rul de raze este mai mare, cu at\^{a}t este mai mic unghiul solid pe care \^{i}l va acoperi o raz\u{a}. Acest lucru poate afecta timpul de calcul, dar \c{s}i un num\u{a}r prea mic de raze poate genera probleme. 
	\bigskip
	
	Pentru a putea \^{i}n\c{t}elege mai bine traiectul unei raze vom folosi Figura \ref{Fig11}. S \c{s}i M reprezint\u{a} sursa \c{s}i microfonul, iar linia trasat\u{a} este drumul parcurs de raz\u{a} p\^{a}n\u{a} la microfon, drum care presupune 4 reflexii speculare. Modelul de propagare al sunetului ce urmeaz\u{a} s\u{a} fie prezentat \^{i}n aceast\u{a} lucrare va folosi doar reflexii speculare.
	\bigskip
	
	\begin{figure}[!htb]
		\centering
		\includegraphics[width=10cm]{imagini/reflectionExample.png}
		\caption{Traiectul unei raze de la surs\u{a} la microfon}
		\label{Fig11}
	\end{figure}
	
	Atunci c\^{a}nd se dore\c{s}te implementarea unui algoritm de acest gen, exist\u{a} c\^{a}\c{t}iva factori foarte importan\c{t}i c\^{a}nd ne g\^{a}ndim la timpul de calcul, precum: num\u{a}rul de raze, lungimea maxim\u{a} pe care o raz\u{a} o poate avea, num\u{a}rul maxim de reflexii pe care \^{i}l poate avea o raz\u{a}, chiar \c{s}i procesorul pe care \^{i}l are ma\c{s}ina de pe care lucr\u{a}m poate influen\c{t}a semnificativ timpul de calcul.
	\bigskip
	
	Astfel, realizarea unui algoritm care modeleaz\u{a} acustic spa\c{t}iile \^{i}nchise presupune dou\u{a} mari etape: realizarea unui model geometric \c{s}i implementarea unui model fizic. Prima etap\u{a} se ocup\u{a} de definirea dimensiunilor \^{i}nc\u{a}perii, a\c{s}ezarea sursei \c{s}i a microfoanelor \^{i}n spa\c{t}iu, trasarea razelor \c{s}i determinarea razelor intersectate cu microfoanele. A doua etap\u{a} presupune calcularea m\u{a}surilor fizice: intensitatea, presiunea, faza, magnitudinea, timpul \c{s}i func\c{t}ia de r\u{a}spuns la impuls. 
	
	\section{Convolu\c{t}ia sunetului}
	
	Convolu\c{t}ia în domeniul timpului înseamnă că spectrele sunt multiplicate. Prin „multiplicarea” spectrelor înțelegem că orice frecvență care este puternică în ambele semnale va fi foarte puternică în semnalul convolut și invers orice frecvență care este slabă în ambele semnale de intrare va fi slabă în semnalul de ieșire. Convolu\c{t}ia implic\u{a} dou\u{a} func\c{t}ii matematice $f$ \c{s}i $g$ care produc o a treia func\c{t}ie $h$ ce reprezint\u{a} modul \^{i}n care forma uneia este modificat\u{a} de c\u{a}tre cealalt\u{a}.
	\bigskip
	
	Atunci c\^{a}nd vorbim despre convolu\c{t}ie, sursa de sunet este numit\u{a} semnal de intrare, iar fi\c{s}ierul de ie\c{s}ire este r\u{a}spunsul la impuls. Fi\c{s}ierul de r\u{a}spuns la impuls are mereu o lungime fix\u{a}.
	\bigskip
	
	În practică, o aplicare relativ simplă a convoluției este locul în care avem „func\c{t}ia de răspunsul la impuls” al unui spațiu. Acest lucru se obține înregistrând o scurtă explozie a unui semnal de bandă largă pe măsură ce este procesat de caracteristicile reverberante ale spațiului. Cu alte cuvinte, a fost procesat de func\c{t}ia de răspuns în frecvență al spațiului similar cu modul în care ar funcționa acest proces în spațiul real. De fapt, convoluția din acest exemplu este pur și simplu o descriere matematică a ceea ce se întâmplă atunci când orice sunet este ,,colorat" de spațiul acustic în care apare, ceea ce este de fapt adevărat pentru toate sunetele din toate spațiile, cu excepția unei camere anecoice. Sunetul convolut va apărea, de asemenea, la aceeași distanță ca în înregistrarea originală a impulsului. Dacă convolu\c{t}ion\u{a}m un sunet de două ori cu același răspuns la impuls, distanța sa aparentă va fi de două ori mai mare.
	\bigskip
	
	Teorema de bază despre domeniul timpului și domeniul frecvenței este că multiplicarea într-un domeniu este echivalentă cu convoluția din celălalt domeniu.
	\bigskip
	
	În cele din urmă, există o diferență tehnică între \textit{convoluție directă}, care este un proces foarte lent, dat fiind că fiecare eșantion din fiecare semnal trebuie să fie multiplicat cu fiecare eșantion din celălalt semnal. O variant\u{a} mai rapid\u{a} de a face acest lucru este folosirea Transformatei Rapide Fourier si folosirea Inversei Transofrmatei Rapide Fourier.
	\bigskip
	
	\^{I}n practic\u{a}, convolu\c{t}ia se realizeaz\u{a} cel mai adesea prin calculul FFT sau analizele spectrale pentru fi\c{s}ierele de intrare \c{s}i r\u{a}spunsul la impuls, \^{i}nmul\c{t}ind spectrele lor \^{i}mpreun\u{a}. Aceasta se nume\c{s}te  \textit{convolu\c{t}ie rapid\u{a}}.
	
	\section{Transformata Rapid\u{a} Fourier \c{s}i Inversa Transofrmatei Rapide Fourier}
	
	Transformata Fourier Rapid\u{a}(FFT) este un algoritm care calculeaz\u{a} transformata direct\u{a}(DFT) a unei secven\c{t}e sau inversa acesteia(IDFT). Analiza Fourier converte\c{s}te un semnal din domeniul timpului sau al spa\c{t}iului \^{i}n domeniul frecven\c{t}elor \c{s}i invers.
	\bigskip
	
	Transformata Fourier Discret\u{a}(DFT) transform\u{a} o secven\c{t}\u{a} de $N$ numere complexe ${x_n} = x_0, x_1, \dots, x_{N-1}$ \^{i}ntr-o alt\u{a} secven\c{t}\u{a} de numere complexe ${X_k} = X_0, X_1, \dots, X_{N-1}$ \c{s}i are urm\u{a}toarea formul\u{a}\cite{fft}:
	\begin{equation}
		X_k = \sum_{n=0}^{N-1}x_n \cdot e^{-\dfrac{i2\pi}{N}kn}
	\end{equation}
	\bigskip
	
	DFT este o opera\c{t}ie foarte util\u{a}, dar destul de costisitoare, din acest motiv a ap\u{a}rut FFT, care calculeaz\u{a} rapid astfel de transform\u{a}ri factoriz\^{a}nd matricea DFT \^{i}ntr-un produs de factori rari(\^{i}n mare parte zero). \^{I}n acest mod, complexitatea algoritmului este redus\u{a} de la $O(N^2)$ la $O(N\lg N)$. Diferența de viteză poate fi enormă, în special pentru seturile de date lungi, unde $N$ poate de ordinul miilor sau milioanelor. În prezența unei erori de rotunjire, mulți algoritmi FFT sunt mult mai exacți decât evaluarea definiției DFT direct sau indirect.
	\bigskip
	
	Reprezentarea \^{i}n domeniul frecven\c{t}\u{a} presupune descopunerea unui semnal acustic \^{i}n semnale sinusoidale caracterizate prin frecven\c{t}\u{a}, amplitudine \c{s}i faz\u{a}. FFT permite modificarea semnalului prin atenuare/eliminare de frecven\c{t}e - filtrare \^{i}n domeniul frecven\c{t}\u{a}.
	\bigskip
	
	Pentru a atinge performan\c{t}e exist\u{a} mai mul\c{t}i algoritmi care calculeaz\u{a} FFT, iar ace\c{s}ti algoritmi, de obicei, presupun \^{i}mpar\c{t}irea polinomului ini\c{t}ial \^{i}n dou\u{a} polinoame. Polinomul calculat poate fi caracterizat de r\u{a}d\u{a}cinile complexe conjugate prin r\u{a}d\u{a}cini de ordin $n$ ale unit\u{a}\c{t}ii. Acest polinom trebuie evaluat la doar $\dfrac{n}{2}$ r\u{a}d\u{a}cini ale unit\u{a}\c{t}ii. Una dintre condi\c{t}iile necesare pentru a atinge aceste performan\c{t}e de viteze este ca $n$ s\u{a} fie o putere de-a lui 2.
	\bigskip
	
	\begin{figure}[!htb]
		\centering
		\includegraphics[width=15cm]{imagini/fft.png}
		\caption{Exemplu de transformare al semnalului cu ajutorul Transformatei Rapide Fourier}
		\label{Fig14}
	\end{figure}
	
	\^{I}n Figura \ref{Fig14}, putem observa trei semnale $x_1, x_2, x_3$ care \^{i}mpreun\u{a} formeaz\u{a} semnalul $x$ \c{s}i sunt transformate cu ajutorul Transformatei Rapide Fourier pentru a compune semnalul transformat.