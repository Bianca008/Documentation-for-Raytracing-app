\^{I}n zilele noastre calculatorul este folosit \^{i}n multiple arii \c{s}i are ca scop solu\c{t}ionarea sau optimizarea unor probleme. Acesta poate rezolva diferite sarcini, deoarece este programabil, adic\u{a} a fost realizat pentru a putea solu\c{t}iona orice cerere dat\u{a} de un program.
\bigskip

Un program reprezint\u{a} un set de instruc\c{t}iuni pe care calculatorul le \^{i}ndepline\c{s}te pentru a rezolva o problem\u{a}. Pe l\^{a}ng\u{a} procesul de programare \^{i}n sine, se reg\u{a}sesc \c{s}i alte procese precum testarea, depanarea \c{s}i mentenan\c{t}a codului surs\u{a} care, asigur\u{a} astfel o calitate superioar\u{a} a codului surs\u{a}. Aplica\c{t}ia final\u{a} trebuie s\u{a} \^{i}ndeplineasc\u{a} o serie de propriet\u{a}\c{t}i fundamentale, indiferent de limbajul de programare ales.
\bigskip

O parte din aceste propriet\u{a}\c{t}i sunt:

\begin{enumerate}
	\utb \textit{Fiabilitatea:} proprietate care reprezint\u{a} corectitudinea programului, adic\u{a} \^{i}n ce m\u{a}sur\u{a} aplica\c{t}ia \^{i}ndepline\c{s}te scopul pentru care a fost conceput\u{a}, depinz\^{a}nd de factori externi precum corectitudinea algoritmilor \c{s}i de cuantumul de erori care pot ap\u{a}rea \^{i}n timpul execu\c{t}iei programului. Mai exact, aceasta poate fi v\u{a}zut\u{a} ca o probabilitate.
	
	\utb \textit{Robuste\c{t}ea:} proprietate care define\c{s}te \^{i}n ce m\u{a}sur\u{a} programul soft reacţioneaz\u{a} la evenimente mai pu\c{t}in a\c{s}teptate cum ar fi accesarea datelor indisponibile sau a unei zone de memorie nealocat\u{a}, introducere de date eronate, etc.
	
	\utb \textit{Uzabilitate:} proprietate care vizeaz\u{a} \^{i}n mod direct utilizatorul \c{s}i se refer\u{a} la u\c{s}urin\c{t}a cu care acesta \^{i}şi poate rezolva problemele prin intermediul aplica\c{t}iei dezvoltate.
	
	\utb \textit{Portabilitate:} proprietate care define\c{s}te multitudinea de platforme \c{s}i sisteme de operare pe care poate rula aplica\c{t}ia dezvoltat\u{a}. De asemenea, se refer\u{a} \c{s}i la uşurin\c{t}a cu care se poate muta codul surs\u{a} de pe o platform\u{a} pe alta.
	
	\utb \textit{Mentenabilitate:} proprietate care eviden\c{t}iaz\u{a} u\c{s}urin\c{t}a de a modifica aplica\c{t}ia, anume prin ad\u{a}ugare de noi funcţionalit\u{a}\c{t}i pentru a satisface noi cerin\c{t}e, fixarea problemelor existente sau adaptarea codului la o versiune actual\u{a} aplica\c{t}iei.
	
	\utb \textit{Eficien\c{t}a/Performan\c{t}a:} proprietate care m\u{a}soar\u{a} resursele de timp \c{s}i spa\c{t}iu de memorie folosite la execu\c{t}ia aplica\c{t}iei. Eficien\c{t}a programului const\u{a} \^{i}n minimizarea resurselor utilizate. Un alt aspect important este gestionarea corect\u{a} a memoriei \c{s}i utilizarea unor algoritmi eficien\c{t}i.
\end{enumerate}

\^{I}n continuare urmeaz\u{a} o prezentare detaliat\u{a} privind tehnologiile folosite \^{i}n cadrul acestei lucr\u{a}ri.

\section{Limbajul de programare C{\#}}

	Un limbaj de programare este un limbaj formal care cuprinde un set de instruc\c{t}iuni care produc diferite tipuri de ie\c{s}ire. Limbajele de programare sunt utilizate în programare pentru a implementa algoritmi.
	\bigskip
	
	Majoritatea limbajelor de programare constau \^{i}n instruc\c{t}iuni pentru calculatoare. Exist\u{a} ma\c{s}ini programabile care folosesc un set de instruc\c{t}iuni specifice, mai degrab\u{a} dec\^{a}t limbaje de programare generale.
	\bigskip
	
	Limbajul de programare C{\#} este un limbaj de programare imperativ, obiect-orientat, asem\u{a}n\u{a}tor sintactic cu Java \c{s}i C++. Acesta a fost creat de Microsoft, ini\c{t}ial \^{i}n cadrul proiectului .NET, la sf\^{a}r\c{s}itul anilor 90, fiind un concurent al limbajului Java. Acestea sunt derivate ale limbajului C$++$. Limbajul C{\#} a fost conceput ca s\u{a} fie simplu, modern,	s\u{a} aib\u{a} un scop general \c{s}i s\u{a} fie orientat pe obiecte.
	\bigskip
	
	Acest limbaj con\c{t}ine dou\u{a} categorii de tipuri de date: \textit{tipuri valoare} \c{s}i \textit{tipuri referin\c{t}\u{a}}. Prima categorie con\c{t}ine tipuri simple, precum: char, int, float, dar \c{s}i tipurile enumerare \c{s}i structur\u{a}, fiind alocate pe stiv\u{a} sau inline \^{i}ntr-o structur\u{a}. Cea de-a doua categorie con\c{t}ine tipurile interfa\c{t}\u{a}, delegat \c{s}i tablou. Un aspect important al limbajului de programare C{\#} este faptul c\u{a} toate tipurile de date sunt derivate direct sau indirect din tipul de date System.Object\cite{limbaj}.
	\bigskip
	
	Astfel, se pot crea aplica\c{t}ii web prin intermediul ASP.NET, aplica\c{t}ii desktop prin WPF(Windows Presentation Foundation) sau aplica\c{t}ii mobile pe Windows Phone. Common Language Runtime(CLR) gestioneaz\u{a} execu\c{t}ia programelor .NET, fiind un Virtual Machine(VM) care ruleaz\u{a} Intermediate Language(IL) \c{s}i ofer\u{a} multiple servicii, precum gestionarea memoriei, securitate, gestionare de excep\c{t}ii, garbage collector, dar \c{s}i altele. 
	\bigskip
	
	\^{I}n cadrul acestui studiu, limbajul de programare a fost folosit pentru a realiza implementarea modelului acustic, iar rezultatele acestuia au fost ilustrate cu ajutorul platformei Unity, dar \c{s}i cu ajutorul limbajului de programare Python.
	
\section{Limbajul de programare Python \c{s}i biblioteca matplotlib}

	Python este un limbaj de programare interpretat, obiect orientat, de nivel înalt, cu semantică dinamică. Structurile sale de date încorporate la nivel înalt, combinate cu tastarea dinamică și legarea dinamică, îl fac foarte atractiv pentru dezvoltarea rapidă a aplicațiilor, precum și pentru a fi utilizat ca limbaj de scriptare sau lipici pentru a conecta componentele existente împreună. Sintaxa simplă, ușor de învățat accentuează lizibilitatea și, prin urmare, reduce costul întreținerii programului. Python acceptă module și pachete, ceea ce încurajează modularitatea programului și reutilizarea codului.
	\bigskip
	
	Acest studiu folose\c{s}te biblioteca matplotlib pentru a reprezenta rezultate comparative pentru func\c{t}ia de r\u{a}spuns \^{i}n timp sau sunetul dup\u{a} convolu\c{t}ie pe mai multe microfoane. Aceasta este o bibliotecă ultil\u{a} pentru crearea de vizualizări statice, animate și interactive în Python. Matplotlib poate fi utilizat în scripturi Python, shell-urile Python și IPython, servere de aplicații web și diverse seturi de instrumente grafice de interfață cu utilizatorul\cite{python}.
	
\section{Platforma Unity}

	O platform\u{a}(IDE- Integrated Development Environment) este o aplica\c{t}ie software care ofer\u{a} facilit\u{a}\c{t}i programatorilor pentru dezvoltarea de soft. O platform\u{a} este alc\u{a}tuit\u{a} din cel pu\c{t}in un editor de cod surs\u{a}, instrumente de automatizare a construc\c{t}iilor \c{s}i un depanator.
	\bigskip
	
	Un singur program \^{i}n care se realizeaz\u{a} dezvoltarea unui soft reprezint\u{a} o platform\u{a}(IDE). Aceast\u{a} platform\u{a} ofer\u{a} multe caracteristici pentru autorizare, modificare, compilare, implementare \c{s}i depanare a produsului soft. Unele platforme sunt specializate pe un limbaj de programare specific, oferind un set de caracteristici care se potrivesc cu paradigma de programare a acelui limbaj. Cu toate acestea, exist\u{a} multe IDE-uri  care suport\u{a} mai multe limbaje de programare.
	\bigskip
	
	Platforma Unity este folosit\u{a} \^{i}n general pentru a crea jocuri \c{s}i poate rula pe mai multe platforme. A fost dezvoltat de Unity Technologies \c{s}i lansat \^{i}n 2005 la Apple Inc's Worldwide Developers Conference ca fiind un game engine exclusiv pentru macOS. 
	\bigskip
	
	Unity este principala platform\u{a} mondial\u{a} pentru crearea \c{s}i operarea de con\c{t}inut 3D interactiv, \^{i}n timp real, oferind instrumente pentru a crea jocuri \c{s}i pentru a le publica pe o gam\u{a} larg\u{a} de dispozitive. Platforma de baz\u{a} Unity permite echipelor creative s\u{a} fie mai productive \^{i}mpreun\u{a}.
	\bigskip
	
	De-a lungul anilor, aceast\u{a} platform\u{a} s-a dezvoltat reu\c{s}ind ast\u{a}zi s\u{a} sus\c{t}in\u{a} peste 25 de platforme. Platforma poate fi utilizat\u{a} pentru a crea jocuri 2D \c{s}i 3D, realitate virtual\u{a} \c{s}i realitate augmentat\u{a}. Unity este utilizat nu numai pentru jocuri video, c\^{a}t \c{s}i \^{i}n domeniul filmelor, arhitecturii, ingineriei si construc\c{t}iilor\cite{unity}.
	
\section{Platforma Blender}
	
	Blender este o aplica\c{t}ie software de grafic\u{a} 3D folosit\u{a} pentru crearea de informa\c{t}ii, efecte vizuale, art\u{a}, modele de tip\u{a}rire 3D, grafic\u{a} de mi\c{s}care, aplica\c{t}ii 3D interactive \c{s}i jocuri pe calculator. Aceasta este cross-platform \c{s}i permite map\u{a}ri UV, folosirea materialelor, a shaderelor, a mesh-urilor f\u{a}r\u{a} probleme.
	\bigskip
	
	\^{I}n acest studiu, programul a fost folosit pentru a putea realiza \^{i}nc\u{a}perea sferic\u{a} din aplica\c{t}ie, \^{i}ntruc\^{a}t era nevoie de un soft ce trebuia s\u{a} permit\u{a} crearea unei fi\c{s}ier de tipul ,,obj'' care s\u{a} con\c{t}in\u{a} o sfer\u{a} cu normalele inversate pentru a putea plasa alte elemente \^{i}n\u{a}untrul acesteia.
	
\section{Biblioteca NWaves, XCharts \c{s}i StandaloneFileBrowser}

	Biblioteca NWaves a fost ini\c{t}ial destinat\u{a} cercet\u{a}rii, vizualiz\u{a}rii \c{s}i pred\u{a}rii elementelor de baz\u{a} ale program\u{a}rii DSP(Digital Signal Processing) \c{s}i a sunetului. To\c{t}i algoritmii sunt implementa\c{t}i \^{i}n limbajul de programare C{\#}, c\^{a}t mai simplu posibil \c{s}i au fost proiecta\c{t}i \^{i}n principal pentru procesarea offline(\^{i}n prezent exist\u{a} \c{s}i multe metode online)\cite{nwaves}.
	\bigskip
	
	Aceast\u{a} bibliotec\u{a} este una open source destinat\u{a} proces\u{a}rii digitale a semnalelor audio, \^{i}nglob\^{a}nd multiple func\c{t}ionalit\u{a}\c{t}i pentru acest domeniu. \^{I}n cadrul studiului, biblioteca a fost folosit\u{a} pentru citirea/creearea fi\c{s}ierelor cu extensia \textit{,,wav"}, dar \c{s}i pentru convolu\c{t}ia sunetelor.
	\bigskip

	XCharts este o bibliotec\u{a} puternic\u{a}, u\c{s}or de utilizat \c{s}i de configurat pentru vizualizarea datelor \^{i}n contextul platformei Unity. Acest proiect a fost dezvoltat sub Unity 2017 \c{s}i .NET 3.5\cite{xcharts}.
	\bigskip
	
	Diferite componente \c{s}i date pot fi  combinate \^{i}n diferite tipuri de diagrame. O component\u{a} XCharts este \^{i}mp\u{a}r\c{t}it\u{a} \^{i}ntr-o component\u{a} principal\u{a} \c{s}i \^{i}n una sau mai multe componente secundare, unde cea principal\u{a} con\c{t}ine toate componentele secundare. 
	\bigskip
	
	Biblioteca a fost folosit\u{a} \^{i}n contextul acestei lucr\u{a}ri pentru a ilustra date statistice privind sunetul, precum: func\c{t}ia de r\u{a}spuns la impuls \^{i}n frecven\c{t}\u{a} \c{s}i \^{i}n timp, magnitudinea, faza, timpul, dar \c{s}i altele.
	\bigskip

	Biblioteca StandaloneFileBrowser este una cross-platform ce permite lucrul cu fi\c{s}iere pentru platforma Unity\cite{standalone}. Aceasta a fost folosit\u{a} \^{i}n realizarea aplica\c{t}iei pentru a permite deschiderea \c{s}i utilizarea fi\c{s}ierelor \textit{,,wav"} pentru a putea crea input-ul pentru modelul acustic.
	

	