	\thispagestyle{empty}
	\begin{center}
		\bf{ABSTRACT}
	\end{center}
	
	
	Această lucrare este organizată în jurul problemei propagării sunetului în contextul spațiilor interioare folosind metoda Ray-Tracing. În zilele noastre, această problemă este adesea întâlnită atât în mediul academic, cât
	și în situații din viața reală, regăsindu-se în domenii precum industria ingineriei acustice și industria jocurilor. Acest studiu oferă un model acustic care sprijină inginerii să construiască în mod corespunzător spații precum amfiteatre, săli de concerte, biserici, moschee, fabrici și multe altele. Mai mult de atât, fiind un subiect de actualitate, industria jocurilor, care a început să capete un rol foarte important în viețile noastre, a început să folosească diferite tehnici pentru propagarea sunetului pentru a imita cât mai bine realitatea și pentru a îmbunătăți calitatea jocurilor. Prin această lucrare se dorește respectarea nevoilor oamenilor de a putea crea încăperi în care sunetul să fie auzit peste tot, de a putea ajuta omul să își mențină sănătatea urechii prin construirea unor încăperi aflate în parametrii acustici, dar și pentru a îmbunătăți experiența omului privind calitatea jocurilor. În această lucrare au fost realizate o serie de experimente diferite pentru a determina parametrii adecvați, precum: care ar fi cel mai potrivit material pentru pereții camerei, care ar trebui să fie frecvența maximă a sunetului, care ar trebui să fie dimensiunile camerei, care ar trebui să fie numărul de raze distribuit în încăpere, care ar fi distanța maximă pe care o rază ar trebui să o permită, dar și altele.
